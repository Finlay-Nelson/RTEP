A github to demonstrate workings of Real Time Embedded (R\+TE) project at the University of Glasgow.

"Employ E\+MG as a method of user input to a gaming environment”

{\bfseries The aim}

The general aim of this project is to develop a system which detects E\+MG signals from the bicep and tricep muscles and use these signal to innervate movement in a video game. {\bfseries Get your flex on!}

Firstly we aim to hack simple, existing games as a proof of concept. We can then develop our own, or hack more complicated games. The more more stuff going on in a game, the better it must be! No?

{\bfseries Methodology}

The methodology is to use two standard Ag/\+Ag\+Cl electrodes placed 20cm apart on the muscle. The signal from these electrodes is sent through a two stage amplifer, the first stage being the differential stage, the second being a gain stage. The output of the amplifier is sent to an A\+DC and then passed to a Raspberry Pi via I2C data protocol for post-\/processing and game connection.

{\bfseries Potential Uses}

The primary use of this system is to increase the submersion effect of the player into a virtual envinroment.

The system also has uses in the field of rehabilitation, and could be used to encourage otherwise unpleasent rehabilitation programs.

This system could be used in combination with F\+ES stimulation to assist those with muscular or neuronal pathologies.

{\bfseries Progress}

We\textquotesingle{}ve achieved a working A\+D7705 pcb to read in a stream of data to start plotting whilst the E\+MG pcb is being fabricated and tested.

A realtime Qt plot has been acheived, the basis of the plot is presented with window.\+cpp.

The E\+MG adc has been selected as is getting implemented onto a custom pcb for 4 channel E\+MG recording. 